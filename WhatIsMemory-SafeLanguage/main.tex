\documentclass{article}
\usepackage[utf8]{inputenc}

\title{What is a Memory-Safe Language? A Breakdown of Rust}
\author{Author}
\date{\today}

\begin{document}

\maketitle

\begin{abstract}
Memory safety is a fundamental requirement for reliable software. This paper introduces the concept of memory-safe programming languages, exploring what features make a language memory-safe and why Rust is a key example. We outline Rust's ownership model, borrow checker, and other safety features, offering a beginner-friendly overview.
\end{abstract}

\section{Introduction}
\label{sec:introduction}
Understanding memory safety is essential for building robust systems. This paper breaks down the basic principles of memory-safe languages and explains how Rust applies these principles in practice.

\section{What is Memory Safety?}
\label{sec:what-is-memory-safety}
A short overview of memory safety, common pitfalls such as buffer overflows and use-after-free, and how languages attempt to address them.

\section{Rust's Approach}
\label{sec:rust-approach}
Summary of Rust's core safety mechanisms: ownership, borrowing, lifetimes, and how they provide guarantees at compile time.

\section{Comparison with Other Languages}
\label{sec:comparison}
Brief notes comparing Rust with C/C++, Java, and other languages with respect to memory management and safety.

\section{Conclusion}
\label{sec:conclusion}
A short closing summary discussing the importance of memory safety and the role Rust plays in modern software development.

\end{document}
